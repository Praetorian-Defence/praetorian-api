\renewcommand\chaptername{Príloha}

\chapter{Používateľská príručka}

\renewcommand*{\thepage}{C\arabic{page}}
\setcounter{page}{1}

Príručka obsahuje informácie o postupe inštalácie systému, jeho konfigurácie a používania.

Celý systém je rozdelený na päť častí, ktoré sa nachádzajú v priloženom elektronickom médiu.
Riešenia \inlinecode{praetorian-api-client} a \inlinecode{praetorian-fabric} slúžia, ako knižnice používané v iných súčastiach systému.
Tým pádom ich nie je potrebné inštalovať, konfigurovať ani spúšťať.
Napriek tomu sa dodatočné informácie o daných knižniciach nachádzajú v súbore \emph{README.md} sídliacom v koreňovom adresári projektu.

Čo sa týka projektu \inlinecode{praetorian-test-deploy}, ten slúži na testovacie účely, ako forma vyvýjaného produktu
obsahujúceho potrebné skripty na komunikáciu s koncovými zariadeniami.

Z týchto dôvodov je príručka podľa dvoch hlavných komponentov systému, čo sú \inlinecode{praetorian-api} a \inlinecode{praetorian-ssh-proxy} rozdelená na iba dve časti.


\section{Praetorian API}

Projekt slúži, ako api server komunikujúci so zariadením používateľa prostredníctvom HTTPS REST komunikácie.
Riešenie obsahuje správu a manažment všetkých entít v systéme.

\subsection{Inštaláca}

Na spustenie daného riešenia je potrebné si nainštalovať:

\begin{enumerate}
\item Programovací jazyk \inlinecode{python 3.7+}:
\begin{itemize}
\item \emph{https://www.python.org/downloads/}
\end{itemize}
\item Relačnú databázu \inlinecode{PostgreSql 12}:
\begin{itemize}
\item \emph{https://www.postgresql.org/download/}
\end{itemize}
\end{enumerate}

Ďalším krokom je vytvorenie virtuálneho prostredia a inštalácia package manažéra \inlinecode{poetry} pre konkrétny projekt pomocou príkazov:
\begin{itemize}
\item \inlinecode{python -m venv venv}
\item \inlinecode{pip install poetry}
\end{itemize}

Posledným krokom inštalácie je inštalácia závislostí pomocou príkazu:
\begin{itemize}
\item \inlinecode{poetry install}
\end{itemize}

\subsection{Konfigurácia}

Pre nastavenie pripojenia databázy PostgreSQL je potrebné vytvoriť súbor \inlinecode{.env}, kde budú nastavené environment
premenné podľa ukážky v súbore \inlinecode{.env.example}.
\newpage

Inicializácia databázy je vykonaná nasledujúcimi príkazmi:
\begin{itemize}
\item \inlinecode{python manage.py migrate}
\item \inlinecode{python manage.py sync\_roles}
\end{itemize}

Na vytvorenie prvého používateľa v systéme s administrátorskými právami je potrebné vykonať nasledujúci príkaz:
\begin{itemize}
\item \inlinecode{python manage.py createsuperuser}
\end{itemize}

\subsection{Návod}

Projekt je možné spustiť príkazom:
\begin{itemize}
\item \inlinecode{python manage.py runserver [HOST]:[PORT]}
\item Príklad: \inlinecode{python manage.py runserver localhost:8000}
\end{itemize}

Pre volanie požiadaviek na spustený api server je potrebné si nainštalovať jednu z platform pre vývoj API, ako napríklad:

\begin{enumerate}
\item \inlinecode{Postman}:
\begin{itemize}
\item \emph{https://www.postman.com/downloads/}
\end{itemize}
\end{enumerate}

Celá REST API dokumentácia vo formáte PDF s príkladmi požiadaviek a odpovedí sa nachádza v obsahu priloženého média v
adresári \emph{docs}.

\section{Praetorian SSH Proxy}

Daný projekt slúži, ako tretia strana zastrešujúca komunikáciu medzi zariadením používateľa a koncovým zariadením zákazníka.
Komunikácia prebieha aj medzi proxy serverom a api serverom pomocou api klienta.

\subsection{Inštaláca}

Na spustenie daného riešenia je potrebné si nainštalovať:

\begin{enumerate}
\item Programovací jazyk \inlinecode{python 3.8+}:
\begin{itemize}
\item \emph{https://www.python.org/downloads/}
\end{itemize}
\end{enumerate}

Ďalším krokom je vytvorenie virtuálneho prostredia a inštalácia package manažéra \inlinecode{poetry} pre konkrétny projekt pomocou príkazov:
\begin{itemize}
\item \inlinecode{python -m venv venv}
\item \inlinecode{pip install poetry}
\end{itemize}

Posledným krokom inštalácie je inštalácia závislostí pomocou príkazu:
\begin{itemize}
\item \inlinecode{poetry install}
\end{itemize}

\subsection{Konfigurácia}

Pre nastavenie pripojenia na api server je potrebné vytvoriť súbor \inlinecode{.env}, kde budú nastavené environment
premenné podľa ukážky v súbore \inlinecode{.env.example}.

\subsection{Návod}

Projekt je možné spustiť príkazom:
\begin{itemize}
\item \inlinecode{python run.py [HOST] [PORT]}
\item Príklad: \inlinecode{python run.py localhost 22}
\end{itemize}

\newpage

Na koncové zariadenie pomocou interaktívneho session je možné sa pripojiť príkazom:
\begin{enumerate}
    \item Nepriamy spôsob:
    \begin{itemize}
        \item \inlinecode{ssh [USERNAME]@[HOST] -p [PORT]}
        \item Príklad: \inlinecode{ssh user@praetorian.sk@localhost -p 22}
    \end{itemize}
    \item Priamy spôsob:
    \begin{itemize}
        \item Priame: \inlinecode{ssh [USERNAME]+[PROJECT]-[REMOTE]@localhost -p 22}
        \item Príklad: \inlinecode{ssh user@praetorian.sk+Project-Remote@localhost -p 22}
    \end{itemize}
\end{enumerate}

Na pripojenie sa ku koncovému zariadeniu pomocou neinteraktívneho session je potrebný projekt s nainštalovanou rozšírenou
knižnicou fabric.
Pre otestovanie tohto spôsobu je možné otvoriť spomínaný \inlinecode{praetorian-test-deploy} projekt v ktorého README.md súbore
sa nachádzajú všetky potrebné informácie.
Po vytvorení skriptu je možné zavolať neinteraktívny session príkazom:

\begin{itemize}
\item \inlinecode{fab [FUNCTION] [REMOTE]}
\item Príklad: \inlinecode{fab deploy production}
\end{itemize}
