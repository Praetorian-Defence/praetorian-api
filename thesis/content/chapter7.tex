\chapter{Záver}\label{ch:zaver}

Prvotnou myšlienkou nášho riešenia bola predovšetkým schopnosť ukladať citlivé firemné údaje na centralizované dátové úložisko
s cieľom využitia potenciálu na automatizovanie procesov, ktoré sú voči zneužitiu z pohľadu informačnej bezpečnosti
najcitlivejšie.
Prístup k citlivým údajom podlieha určitým oprávneniam, ktoré je možné nakonfigurovať pre každú z firiem rozdielne, aby
odzrkadľovali vnútro-firemnú štruktúru.
Z dôvodu vyžadovania takejto formy prístupu u všetkých typoch firemných procesov, sme sa rozhodli využiť danú myšlienku
centralizovaného úložiska rozšírením vo forme nadstavby api rozhrania, cez ktoré bude umožnená bezpečná komunikácia s centralizovaným
úložiskom.
Týmto riešením je umožnené konfigurovať nielen vnútro-firemnú štruktúru, no taktiež manažovanie citlivých údajov a ich monitorovanie.
Keďže výsledné webové api rozhranie reprezentovalo samostatný modul, bolo možné vytvoriť ďalšie moduly s rôznymi funkcionalitami.
Modul vo forme ssh proxy servera vie s webovým api rozhraním komunikovať a poskytuje ďalšie možné formy nadstavby,
ako napríklad poskytnutie komunikácie so vzdialenými zariadeniami.
Ďalším rozšírením bolo zabezpečenie komunikácie implementáciou api klienta, ktorý bol všestranne využívaný naprieč rozličnými
modulmi riešenia.
Posledným rozšírením bola nadstavba knižnice fabric, pre nami špecifikovanú komunikáciu, ktorá slúžila ako ukážka rozšírenia
ktoréhokoľvek nástroja tohto typu bez ohľadu na použitý programovací jazyk, alebo technológiu.
Kľúčovým bodom výskumu bolo navrhnutie a implementácia komunikácie medzi dvoma uzlami, ktoré sú medzi sebou anonymizované za
pomoci tretej strany.
Medzi zariadeniami by bolo možné posielať rôzne citlivé, prístupové a konfiguračné údaje bez nutnosti používateľa k nim
akýmkoľvek spôsobom pristúpiť, pričom všetky tieto operácie museli podliehať striktnou kontrolou bezpečnosti, aby bolo
dané riešenie možné využívať v praxi softvérovými firmami.
Systém bol navrhovaný a implementovaný dodržiavaním kritérií, ako bezpečnosť, modularita, udržiavateľnosť, rozšíriteľnosť
a auditovateľnosť.

Jedným z problémov, ktorý sa počas implementácie ssh proxy servera vyskytol, nastával pri odchytávaní príkazov v interaktívnom
móde pripojenia vzhľadom na whitelist, alebo blacklist proces filtrovania.
Jadro problému spočívalo v spôsobe implementácie, ktorá fungovala iba spôsobom presmerovania vstupov a výstupov komunikujúcich
uzlov.
Tým pádom boli z klienta na vzdialené zariadenie preposielané všetky znaky bez možnosti buffrovania vstupu, následného
rozoznania jednotlivých príkazov a následnej kontroly pred ich odoslaním na vzdialené zariadenie.

\section{Vízia}\label{sec:vizia}

Existuje značný počet možností, ktorými je dané riešenie možné rozširovať a upravovať jeho stávajúce funkcionality.
Medzi najvyššie priority patrí používateľské grafické rozhranie pre správu jednotlivých častí systému.
Rozdeľovalo by sa na dve časti, kde prvou by bolo administrátorské rozhranie spravujúce tie najviac dôverné súčasti systému,
ako napríklad: Prístupové údaje, pridávanie projektov, asociácia koncových zariadení do projektov správa práv používateľov a iné.
Druhou časťou by bolo rozhranie zamestnancov, ktorým by bolo umožnené prehliadať asociované projekty, koncové zariadenia, možnosti
prístupov a mieru operácií, ktoré môžu na danom projekte, alebo zariadení vykonať.
Grafické rozhranie by mohlo byť implementované v podobe webovej stránky, alebo mobilnej aplikácie.
Čo sa týka úpravy stávajúcej funkcionality, anonymizovaná komunikácia by mohla byť využitá na dynamickú zmenu prístupových údajov
na koncové zariadenia v určitých časových intervaloch, čím by sa minimalizovalo riziko odchytenia a dešifrovania aktuálnych
prístupových údajov pomocou MITM útoku.
Pre ssh proxy server by bolo potrebné rozšíriť formu autentifikácie pomocou RSA kľúčov a upraviť kontrolu zariadení používateľov
pomocou SSL certifikátov.
Autorizácia by sa mohla rozšíriť o kontrolu prístupu používateľov k jednotlivým koncovým zariadeniam pre jednotlivé projekty,
čo bolo spomínané aj v štvrtej kapitole: Návrh riešenia.
Pre zvýšenie miery kontroly a bezpečnosti je možné pridanie overenia druhu siete, z ktorej sa používateľ pripája a tým zabrániť
pripojenie z verejných sietí mimo lokálnej siete firmy, alebo vpn.
Autentifikáciu je možné rozšíriť pre podporu LDAP rozhrania.
Počas komunikácie medzi zariadením používateľa a koncovým zariadením zákazníka by ssh proxy server mohol filtrovať nežiadúce príkazy
pomocou metódy whitelist, alebo blacklist.
