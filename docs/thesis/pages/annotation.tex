%%%%%%%%%%%%%%%%%%%%%%%%%%%%%%%%%%%%%%%%%%%%%%%%%%%%%%%%%%%%%%%%%%%%%%%%%%%%%%%%%%%%%%%%
%%
%% Slovak annotation
%%
%%%%%%%%%%%%%%%%%%%%%%%%%%%%%%%%%%%%%%%%%%%%%%%%%%%%%%%%%%%%%%%%%%%%%%%%%%%%%%%%%%%%%%%%
\newpage
\thispagestyle{plain}
\begin{flushleft}
\begin{Large}
\textbf{Anotácia} \\
\end{Large}
Slovenská technická univerzita v Bratislave \\
FAKULTA INFORMATIKY A INFORMAČNÝCH TECHNOLÓGIÍ\vspace{0.6cm}
\noindent
	\textrm{
	    \begin{normalsize}
	    \begin{tabular}{@{}ll@{}}
			Študijný program: & \FIITstudyProgramSK\\
			Diplomová práca: & \FIITtitleSK \\
			Autor: & \FIITauthor \\
			Vedúci práce: & \FIITsupervisor \\
			Jún,\ 2020
	    \end{tabular}
	  	\end{normalsize}
	}
\noindent
\end{flushleft}

Každá softvérová firma musí riešiť otázky, ktoré vyplývajú z problémov ohľadom informačnej bezpečnosti.
Takéto typy problémov by nemali byť prehliadané, z dôvodu možných následkov v podobe výskytu bezpečnostných rizík.
Obsah práce sa v prvom rade zameriava na analýzu ich možných výskytov vo firemnom prostredí, s možnosťami ich následného
odstránenia, prípadne minimalizovania podľa overených bezpečnostných riešení používaných v súčasnosti.
Tieto riešenia sú následne rozobrané z pohľadu manažmentu informačnej bezpečnosti a prislúchajúcim štandardom.
Výstupom práce je softvérový produkt, ktorý bude využívať súčasné a overené metódy na odstránenie alebo minimalizovanie
daných rizík v praxi.
Produkt sa bude zameriavať na autorizáciu, ktorá bude vyplývať z presne stanovenej hierarchie softvérovej firmy.
Dôležitou súčasťou je taktiež správna a bezpečná autentifikácia spolu so správnym ukladaním a manažovaním citlivých údajov.
Tieto údaje budú slúžiť ako prístupové kľúče ku koncovým zariadeniam zákazníkov, pričom cieľom je na tieto zariadenia
bezpečne a automatizovane nasadiť finálny produkt.
Z tohto pohľadu je dôležité brať do úvahy aj možné hrozby, ktorými môžu byť kompromitovaní aj samotní zákazníci.
Z tohto dôvodu je umožnený iba dočasný prístup ku koncovým zariadeniam, kde samotné prístupové údaje budú dynamicky
menené na oboch stranách komunikujúcich uzlov.
\emptypage

%%%%%%%%%%%%%%%%%%%%%%%%%%%%%%%%%%%%%%%%%%%%%%%%%%%%%%%%%%%%%%%%%%%%%%%%%%%%%%%%%%%%%%%%
%%
%% English annotation
%%
%%%%%%%%%%%%%%%%%%%%%%%%%%%%%%%%%%%%%%%%%%%%%%%%%%%%%%%%%%%%%%%%%%%%%%%%%%%%%%%%%%%%%%%%
\newpage
\thispagestyle{plain}
\begin{flushleft}
\begin{Large}
\textbf{Annotation} \\
\end{Large}
Slovak University of Technology Bratislava \\
FACULTY OF INFORMATICS AND INFORMATION TECHNOLOGIES\vspace{0.6cm}
\noindent
	\textrm{
	    \begin{normalsize}
	    \begin{tabular}{@{}ll@{}}
			Study program: & \FIITstudyProgram\\
			Master thesis: & \FIITtitle \\
			Author: & \FIITauthor \\
			Supervisor: & \FIITsupervisor \\
			June,\ 2020
	    \end{tabular}
	  	\end{normalsize}
	}
\noindent
\end{flushleft}
Every software company has to deal with issues that arise from information security issues.
These types of problems should not be overlooked, due to the possible consequences in terms of security risks.
The content of the work is primarily focused on the analysis of their possible occurrences in the corporate environment,
with the possibility of their subsequent removal, or minimization according to proven security solutions currently used.
These solutions are then analyzed from the perspective of information security management and the relevant standards.
The output of the work is a software product that will use current and proven methods to eliminate or minimize the risks in practice.
The product will focus on authorization, which will result from a well-defined hierarchy of the software company.
Correct and secure authentication is also an important part, along with the correct storage and management of sensitive data.
This data will serve as access keys to customers end devices, with the goal of securely and automatically deploying
the final product to those devices.
From this point of view, it is important to take into account the possible threats that may compromise customers themselves.
For this reason, only temporary access to the terminal devices is allowed, where the access data itself will be dynamically
changed on both sides of the communicating nodes.
\emptypage
