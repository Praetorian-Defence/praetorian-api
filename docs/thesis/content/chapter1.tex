\chapter{Úvod}\label{ch:úvod}

\par V súčasnej dobe je pre väčšinu informatikov dostupná veľká škála možností, akým spôsobom sa vedia v praxi angažovať.
Jedným z populárnych spôsobov je možnosť založenia vlastnej softvérovej firmy. V posledných rokoch počet takýchto firiem
prudko narástol, z čoho väčšina z nich sú malé, alebo mikro podniky~\cite{SoftwareBusiness}. Veľký počet takýchto firiem musí zniesť
nadmernú zodpovednosť v podobe veľkých projektov, ktoré vyžadujú profesionálny prístup. Z tohto hľadiska je veľmi dôležité
dbať ohľad na bezpečnosť firmy a hlavne aj jej zákazníkov. Práve z hľadiska informačnej bezpečnosti~\cite{IB} vzniká v softvérových
firmách veľa otázok súvisiacich s bezpečnostnými problémami. Prehliadanie, alebo ignorovanie takýchto výstražných znamení
najmä v spomínaných firmách s menším vzrastom má vo väčšine prípadov za následok vznik bezpečnostných rizík. Takéto riziká
je možné z pohľadu potenciálneho útočníka zneužiť, čím vie softvérovej firme, v určitých prípadoch aj jej zákazníkom napáchať
nemalé škody. Takéto typy škôd môžu mať za následok stratu finančných prostriedkov, zvýšiť čas potrebný na dodanie finálneho
produktu, poškodiť povesť firmy, prípadne kompromitovať koncové zariadenia zákazníkov. Čo má teda za následok prehliadanie
jednotlivých bezpečnostných problémov u väčšiny softvérových firiem? Aké sú v súčasnej dobe možnosti dané problémy v praxi
riešiť v súlade s manažmentom informačnej bezpečnosti? V akých prípadoch sa dané problémy vyskytujú a aké konkrétne riziká
z nich vyplývajú? V prvom rade našou úlohou bude odhaliť veľkú časť hrozieb zaoberajúcich sa predovšetkým oblasťou
"kontaktu" medzi softvérovou firmou a zákazníkom. Daná oblasť je podmienená presne navrhnutými a špecifikovanými bezpečnostnými
postupmi, ktoré sú nevyhnutné pre správnu a bezpečnú manipuláciu s citlivými údajmi zákazníkov. Medzi postupy napríklad patrí
presne stanovená hierarchia medzi zamestnancami v internom prostredí firmy v súlade s manažmentom informačnej bezpečnosti.
Z toho vyplíva, že na správnu a bezpečnú manipuláciu s cilivými prostriedkami zákazníkov budú musieť byť presne stanovené
bezpečnostné pravidlá firmy. Náplňou našej práce je teda navrhnúť bezpečnostný informačný systém, ktorý bude dodržiavanie
navrhnutých firemných pravidiel striktne validovať s cieľom bezpečného uloženia firemných citlivých údajov, kde napríklad
patria prístupové údaje ku koncovým zariadeniam zákazníkov. Medzi dôležité bezpečnostné oblasti riešenia bude patriť správna
a bezpečná autentifikácia, autorizácia, ukladanie citlivých údajov a informácií v súlade s GDPR~\cite{GDPR}. Medzi ďalšie témy, ktorými
sa bude naša práca zaoberať sú bezpečné komunikačné protokoly na výmenu prístupových údajov a ich validáciu. Automatizácia
nasadenia finálnych softvérových produktov na koncové zariadenia je taktiež jedným z cieľov našej práce. Vysokou prioritou
riešenia je jeho vysoká modularita~\cite{Modularity} a možná jednoduchá rozšíriteľnosť~\cite{Scalability}.
