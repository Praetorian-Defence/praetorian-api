\chapter{Úvod}\label{ch:úvod}

V súčasnej dobe je pre väčšinu informatikov dostupná veľká škála možností, akým spôsobom sa vedia v praxi angažovať.
Jedným z populárnych spôsobov je možnosť založenia vlastnej softvérovej firmy.
S narastajúcou potrebou využívania softvérových služieb a digitalizácie sa taktiež zvýšil počet firiem poskytujúcich tieto riešenia.
Príkladom je Nemecko, kde od roku 2008 po rok 2017 narástol počet softvérových firiem o približne o 24\%~\cite{CompaniesGerman}.
V škótsku bol nárast počtu v rovnakom časovom období až o 60\%~\cite{SoftwareBusiness}, pričom majoritnými typmi nových
firiem sú stredné a malé podniky~\cite{CompaniesScotch}.
Na to, aby softvérová firma zabezpečila so zákazníkom profesionálny a adekvátny prístup, je nutné dodržovať viacero noriem.
Jedným zo základných princípov softvérového inžinierstva sú vlastnosti softvéru, ako napríklad spoľahlivosť alebo integrita.
Na druhej strane korektný postup pri tvorbe softvéru nie je jediný z dôležitých čŕt dobrej a úspešnej firmy.
Medzi ďalšie dôležité aspekty patrí aj vnútro firemné zriadenie.
Z tohto hľadiska je veľmi dôležité dbať ohľad na bezpečnosť firmy a hlavne aj jej zákazníkov.
Práve z hľadiska informačnej bezpečnosti~\cite{IB}, vzniká v softvérových firmách veľa otázok súvisiacich s bezpečnostnými problémami.
Prehliadanie, alebo ignorovanie takýchto výstražných znamení, najmä v spomínaných firmách s menším vzrastom má vo väčšine
prípadov za následok vznik bezpečnostných rizík.
Takéto riziká je možné z pohľadu potenciálneho útočníka zneužiť, čím vie softvérovej firme, v určitých prípadoch aj jej
zákazníkom napáchať nemalé škody.
Takéto typy škôd môžu mať za následok stratu finančných prostriedkov, zvýšiť čas potrebný na dodanie finálneho
produktu, poškodiť povesť firmy, prípadne kompromitovať koncové zariadenia zákazníkov.
Čo má teda za následok prehliadanie jednotlivých bezpečnostných problémov u väčšiny softvérových firiem?
Aké sú v súčasnej dobe možnosti dané problémy v praxi riešiť v súlade s manažmentom informačnej bezpečnosti?
V akých prípadoch sa dané problémy vyskytujú a aké konkrétne riziká z nich vyplývajú?
V prvom rade našou úlohou bude odhaliť veľkú časť hrozieb zaoberajúcich sa predovšetkým oblasťou „kontaktu“ medzi softvérovou firmou a zákazníkom.
Daná oblasť je podmienená presne navrhnutými a špecifikovanými bezpečnostnými
postupmi, ktoré sú nevyhnutné pre správnu a bezpečnú manipuláciu s citlivými údajmi zákazníkov.
Medzi postupy napríklad patrí presne stanovená hierarchia medzi zamestnancami v internom prostredí firmy v súlade s manažmentom informačnej bezpečnosti.
Z toho vyplýva, že na správnu a bezpečnú manipuláciu s citlivými prostriedkami zákazníkov budú musieť byť presne stanovené
bezpečnostné pravidlá firmy.
Náplňou našej práce je teda navrhnúť bezpečnostný informačný systém, ktorý bude dodržiavanie navrhnutých firemných
pravidiel striktne validovať s cieľom bezpečného uloženia firemných citlivých údajov, kde napríklad patria prístupové
údaje ku koncovým zariadeniam zákazníkov.
Medzi dôležité bezpečnostné oblasti riešenia bude patriť správna a bezpečná autentifikácia, autorizácia, ukladanie
citlivých údajov a informácií v súlade s ochranou osobných údajov (GDPR)~\cite{GDPR}.
Medzi ďalšie témy, ktorými sa bude naša práca zaoberať sú bezpečné komunikačné protokoly na výmenu prístupových údajov a ich validáciu.
Automatizácia nasadenia finálnych softvérových produktov na koncové zariadenia je taktiež jedným z cieľov našej práce.
Vysokou prioritou riešenia je jeho vysoká modularita~\cite{Modularity} a možná jednoduchá rozšíriteľnosť~\cite{Scalability}.
