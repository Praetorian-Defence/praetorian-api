\chapter{Zhodnotenie}\label{ch:zhodnotenie}

Prvotnou myšlienkou nášho riešenia bola predovšetkým schopnosť ukladať citlivé firemné údaje na centralizované dátové úložisko
s cieľom využitia potenciálu na automatizovanie procesov, ktoré sú voči zneužitiu z pohľadu informačnej bezpečnosti
najcitlivejšie.
Prístup k citlivým údajom podlieha určitým oprávneniam, ktoré je možné nakonfigurovať pre každú z firiem rozdielne, aby
odzrkadľovali vnútro-firemnú štruktúru.
Z dôvodu vyžadovania takejto formy prístupu u všetkých typoch firemných procesov, sme sa rozhodli využiť danú myšlienku
centralizovaného úložiska rozšírením vo forme nadstavby api rozhrania, cez ktoré bude umožnená bezpečná komunikácia s centralizovaným
úložiskom.
Týmto riešením je umožnené konfigurovať nielen vnútro-firemnú štruktúru, no taktiež manažovanie citlivých údajov a ich monitorovanie.
Keďže výsledné webové api rozhranie reprezentovalo samostatný modul, bolo možné vytvoriť ďalšie moduly s rôznymi funkcionalitami,
ako napríklad zabezpečenie prístupu na vzdialené zariadenia.
Takýto modul vie s webovým api rozhraním komunikovať a poskytuje ďalšie možné formy nadstavby, ako napríklad poskytnutie
komunikácie s vzdialenými zariadeniami.

Jedným z problémov, ktorý sa počas implementácie ssh proxy servera vyskytol, nastával pri odchytávaní príkazov v interaktívnom
móde pripojenia vzhľadom na whitelist, alebo blacklist proces filtrovania.
Jadro problému spočívalo v spôsobe implementácie, ktorá fungovala iba spôsobom presmerovania vstupov a výstupov komunikujúcich
uzlov.
Tým pádom boli z klienta na vzdialené zariadenie preposielané všetky znaky bez možnosti buffrovania vstupu, následného
rozoznania jednotlivých príkazov a následnej kontroly pred ich odoslaním na vzdialené zariadenie.
Druhý problém spočíval v kvantitatívnom vyhodnotení nášho riešenia.
Hlavnými kritériami systému je bezpečnosť, modularita, udržiavateľnosť a rozšíriteľnosť s cieľom jeho použitia softvérovou
firmou v praxi.

Existuje vysoký počet možností, ktorými je dané riešenie možné zlepšovať.
Hlavnou prioritou pri práci v ďalšej fáze projektu bude vyriešenie filtrovania v interaktívnom prostredí ssh, kde jednou
z možných riešení je rozdelenie daného prostredia na dva ďalšie typy, pričom prvý by bol promiskuitný, ktorého úlohou
by bol neobmedzený prístup bez kontroly vstupov.
Druhý by vstup ukladal do vyrovnávacej pamäte dovtedy, dokiaľ nebude poslaný znak nového riadku, alebo dokiaľ neuplynie
časový interval.
Následne by bol pred odoslaním daný vstup skontrolovaný a napokon poslaný na koncové zariadenie.
Ďalšími možnosťami zlepšenia by bolo systémové zaznamenávanie udalostí na ssh proxy serveri do syslogu, alebo dané riešenie
kvantitatívne a kvalitatívne ohodnotiť podľa penetračných testov, alebo softvérových metrík.
