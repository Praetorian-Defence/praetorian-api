\chapter{Analýza}\label{ch:analýza}

\section{Bezpečnostné hrozby}\label{sec:podsekcia-1}
\par Z hľadiska bezpečnosti musí väčšina firiem čeliť viacerým hrozbám. V ideálnom prostredí a vo veľkých softvérových
firmách slúži na analýzu bezpečnostných hrozieb takzvaný Systém riadenia informačnej bezpečnosti
(Information Security Management System - ISMS) podľa normy ISO 27000~\cite{ISO27001:2013}. Úlohou takéhoto systému je
definovanie a spravovanie rizík, ktoré môžu vyplívať z hrozieb a zraniteľností. Dôležitou časťou je ochrana aktív s vysokou
hodnotou. Na obrázku~\ref{obr_1} je možné vidieť schému riadenia takýchto rizík.

\par Medzi najzákladnejšie bezpečnostné problémy napríklad patrí nezabezpečený prístup k jednotlivým projektom medzi zamestnancami.
K jednotlivým projektom v softvérovej firme by mali mať prístup iba tí zamestnanci, ktorí sa podieľajú na jeho správe a implementácii.
Vystavenie prístupových údajov, rôznych informácií o zákazníkovi a jeho používateľoch neprislúchajúcemu zamestnancovi,
môže spôsobiť veľa bezpečnostných rizík. Napríklad zamestnanec môže zneužiť dátové úložisko nachádzajúce sa v produkcii.
Zamestnanec môže týmto spôsobom  napríklad získať citlivé informácie o tržbách zákazníka alebo adrese. Taktiež vie získať
prístupové dáta k jeho koncovým zariadeniam a kompromitovať ich. Týmto bezpečnostným rizikám prislúchajú aj možné následky
nie len na strane softvérovej firmy.

\begin{figure}[H]
\begin{center}\includegraphics[width=\textwidth,height=6cm,keepaspectratio=true]{assets/risks.png}\end{center}
\caption[Prehľadová schéma riadenia rizík]{Prehľadová schéma riadenia rizík~\cite{RiskManagement}.}\label{obr_1}
\end{figure}

\par Daný problém sa môže vyskytovať aj vo forme zlých právomocí k jednotlivým aktívam. Softvérová firma teda môže mať
navrhnutú výbornú izoláciu medzi jednotlivými projekatmi a zamestnancami. Chybou môže byť aj neprimerané privilégium vo
forme nadmerného prístupu k prostriedku. Príkladom môže byť zamestnanec, ktorý nemá vo firme prístup k dátovému úložisku
a jeho následnému upraveniu. Ak nebol braný do úvahy aj prístup na prehliadanie jednotlivých údajov, takýto zamestnanec
síce nevie dáta zmeniť alebo vymazať, no je schopný prehliadať citlivé údaje a ich obsah zneužiť.

\par Medzi ďalšie problémy patrí takzvaná likvidácia dát. Príkladom môže byť odchod zamestnanca z firmy. Buďe mať stále po roku
prístupy na jeho minulé projekty? Bude mu prístup odobraný po ukončení pracovného pomeru? Prípadne, čo ak na osobnom stretnutí
prezradí prístupové údaje od bývalých projektov? Je možné si tieto údaje útočníkom zapísať a následne sa dostať k prostriedkom?
Čo ak sa samotný zamestnanec bude chcieť po nečakanej výpovedi pomstiť napáchaním škôd? V situácii, kedy zamestnanec nemá
zrušené prístupové údaje, pričom bývalý projekt je nasadený a v praxi využívaný sa jedná o veľmi veľké bezpečnostné riziko.

\par Ak sa jedná o manipuláciu so samotnými citlivými a prístupovými údajmi, tie môžu podliehať, ako bolo už spomínané
neautorizovaným osobám. Medzi najzákladnejšie riziká sú nekryptované, alebo nehašované citlivé dáta uložené v surovej forme
na dátovom úložisku. Príkladom využitia danej zraniteľnosti môže byť heslo používateľa. Správca ani vlastník systému by nemal
vedieť prístupové údaje používateľov v systéme. Tu je možná forma zneužitia takýchto informácií k prístupu do používateľovho
konta. U systémov pracujúcich s finančnými prostriedkami, by mohlo ísť o krádež finančných prostriedkov používateľov vo
vlastný prospech.

\par Prístup k vzdialeným zariadeniam zákazníka firmy by mala mať iba osoba, ktorá má na starosti nasadenie softvérového riešenia.
Keďže sa jedná o veľmi zodpovenú rolu vo firme, možnou hrozbou je zneužitie prístupových údajov odchytením pomocou útoku
MITM~\cite{MITM}, prípadne vyzradením údajov tretej strane podľahnutím phishing útoku~\cite{Phishing}.
Je možné sa chrániť aj proti takýmto hrozbám?

\begin{figure}[H]
\begin{center}\includegraphics[width=\textwidth,height=6cm,keepaspectratio=true]{assets/priviledge_abuse.jpg}\end{center}
\caption[Najslabšie bezpečnostné miesta v softvérových firmách]{Najslabšie bezpečnostné miesta v softvérových firmách~\cite{WeaknestLink}.}\label{obr_2}
\end{figure}

\par Ako môžme vidieť na obrázku~\ref{obr_2}, najzraniteľnejším miestom v softvérových firmách je zneužitie práv zamestnancov.

\section{Súčasné opatrenia chrániace pred hrozbami}\label{sec:podsekcia-2}

\par Po vymenovaní konkrétnych hrozieb a možných rizík, ktoré z nich vyplívajú je dôležité zistiť formu súčasnývh riešení
protiopatrení, ktoré sú schopné možné riziká odstrániť, prípadne znížiť ich naplnenie. V praxi je nutné finančne predpovedať
náklady firmy vynaložené na opatrenie a finančne predpovedať náklady možného následku po zneužití rizika. Ak je napáchaná
škoda zo zneužitia nižšia, ako dané protiopatrenie na zamedzenie takejto formy útoku, pre softvérovu firmu a jej zákazníkov
je zamedzenie takéhoto typu hrozby z pohľadu manažmentu neoptimálne. Z takýchto, pre firmu nepodstatných rizík je vytvorený
zoznam takzvaných zvyškových rizík, pričom ich znázornenie môžme vidieť na obrázku~\ref{obr_3}.

\begin{figure}[H]
\begin{center}\includegraphics[width=\textwidth,height=6cm,keepaspectratio=true]{assets/zvyskove_riziko.png}\end{center}
\caption[Vizualizácia zvyškových rizík na základe hodnoty protiopatrení]{Vizualizácia zvyškových rizík na základe hodnoty protiopatrení~\cite{RiadenieRizik}.}\label{obr_3}
\end{figure}

\par Priebeh celého procesu riadenia rizík sa vykonáva v nasledovných krokoch. V prvom rade je nutné identifikovať riziko.
Druhým krokom je na základe druhu rizika vhodne stanoviť druh protiopatrenia. Medzi možné spôsoby patria: Zníženie rizika,
Presun rizika, Vyhnutie sa riziku, alebo zachovanie rizika. Po validácii ošetreného rizika vznikajú podproblémy vo forme
zvyškových rizík. Tieto zvyškové riziká sú akceptované vtedy, ak nákladovosť na ich protiopatrenia sú vyššie, ako ich výška
možných škôd. Kontinuálne vykonávanie procesu riadenia rizík je možné znázorniť aj na upravenom Demingovom cykle (tzv. model PDCA)
z bezpečnostnej normy ISO 27001. Vykonávanie podľa daného modelu sa realizuje v štyroch fázach a to: Plánovať – Vykonávať –
Kontrolovať – Pôsobiť. Takýto model je možné prispôsobiť a upraviť pre potreby procesu riadenia rizík podľa obrázka~\ref{obr_3}
nasledovne.

\begin{figure}[H]
\begin{center}\includegraphics[width=\textwidth,height=6cm,keepaspectratio=true]{assets/pdca.png}\end{center}
\caption[Upravený PDCA model z normy ISO 27001]{Upravený PDCA model z normy ISO 27001~\cite{RiadenieRizik}.}\label{obr_4}
\end{figure}

\par Po stručnom vysvetlení procesu riadenia rizík z pohľadu manažmentu informačnej bezpečnosti je na rade vysvetlenie
jednotlivých hrozieb spomenutých v predchádzajúcej časti práce.

\par Prvou hrozbou bol neautorizovaný prístup k aktívam neoprávnenými
osobami. Daný problém je z veľkej miery ovplivnený vnútornou hierarchiou firmy. Ak firma má v súčasnosti vytvorenú hierarchiu
a každý zamestnanec má svoju špecifickú rolu na danom projekte, nutnosťou pre zamedzenie výskytu rizík vyplívajúcich z
danej hrozby je nasadenie systému autorizácie, ktorý bude jednotlivé roly vo firme validovať, pričom prístup zamietne
tým zamestnancom, ktorých rola nedovoľuje prístup k danému aktívu.

\par Jedným z najrelevantnejších (presne mapujúci danú hierarchiu firmy do systému) autorizačných modelov v súčasnej dobe, je riadenie prístupu na základe rolí (Role Based
Access Control - RBAC) podľa štandardu definovanom v ANSI/INCITS 359–2004~\cite{RBAC}. Daný model patrí k ideológii opatrnej
bezpečnostnej politiky~\cite{OpatrnaBezpecnostnaPolitika}. Podstatou danej ideológie je skutočnosť, že v systéme je zakázané
všetko, čo nie je explicitne povolené. Tým pádom si vieme model RBAC rozdeliť na dve časti. Prvou je zoznam všetkých oprávnení,
ktoré sú v systéme umožnené. Druhou skupinou sú jednotlivé role, ktoré majú podľa ich významu zodpovedajúce oprávnenia.
RBAC implementácia spočíva v štyroch procesoch: Vytvorenie používateľa, vytvorenie role, priradenie role používateľovi,
a priradenie oprávnenia roli. Jedným z problémov nastávajúcich v dynamických softvérových firmách sú výnimky vzhľadom
na určité meniace sa podmienky. Následkom výnimky vznikajú ďalšie role, ktorých počet môže byť časom neudržateľný,
pričom takto zneužité riešenie autorizovaného prístupu má za následok opätovné zvýšenie rizika zneužitia právomocí.
Jedným z možných riešení je vytvorenie dodatočnej tretej časti spočívajúcej vo vzťahu medzi samotným používateľom a
oprávneniami. Takýto vzťah by pre každého používateľa v systéme bol unikátny a nezávislý od jeho role, na rozdieľ od mätúceho
vytvárania rozličných rolí s výnimkami. Ďalšou výhodou takéhoto vzťahu je oddelenie medzi oprávneniami rolí a výnimkovými
oprávneniami pre konkrétnich používateľov.

\par Medzi veĺmi diskutované autorizačné metódy patrí riadenie prístupov na základe atribútov (Attribute-based access
control - ABAC), pričom kĺúčový štandard, z ktorého ABAC vzišiel je XACML~\cite{XACML}. Samotný koncept
sa považuje za autorizačný model "novej generácie". Dôvodom je iný spôsob prístupu k
autorizačnej problematike. Narozdieľ od RBAC udeľovanie prístupu používateľom sa riadi na základe jednotlivých aktív,
inak povedané každej súčasti systému~\cite{ABAC_RBAC_Attributes}. Tento spôsob je veľmi užitočný, práve v spomínaných veľkých korporátnych softvérových
firmách s veľmi výraznou dynamikou riadenia. Pridelenie rolí používateľom teda naďalej nezávisí od ich role, ale od
jednotlivých autorizačných pravidiel vyplívajúcich z namapovaných firemných aktív. Týmto spôsobom je dokonca možné hodnotiť
atribúty subjektov a zdrojov ešte pred ich zavedením do autorizačného systému. Na druhej strane má daný koncept obmedzenie
v konfigurácii, pričom určenie povolení daného používateľa je vzhľadom na návrh metódy veľmi obtiažne. Pre danú bezpečnostnú
problematiku, ktorou sa v práci zaoberáme vznikajú ďalšie nevýhody tohto riešenia a to náročnosť implementácie a bezvýznamnosť danej metódy pre
softvérové firmy menších rozmerov. Na implementáciu daného konceptu je taktiež nutný podrobný návrh firemnej politiky~\cite{RBAC_ABAC_Encryption}.

\par Druhou hrozbou, ktorá so sebou prinášala veľký počet rizík bola otázka likvidácie dát. Jednoduchým teoretickým
riešením je všetky dáta a účty, ktoré už nie sú potrebné vymazať. Realita vo firmách je žial oveľa zložitejšia. Ak zamestnanec
z firmy odíde, neostane po ňom vo väčšine malých firiem iba jedno konto. Tieto kontá a prístupové údaje bude nutné všetky vyhľadať,
pričom sa vyskytujú na rôznych zariadeniach, službách technológiach a podobne. Preto musí byť touto pracnou úlohou poverený
zamestnanec z firmy, ktorému môže celý proces trvať niekoľko minút až hodín. Takýto spôsob likvidácie dát je veľmi časovo náročný
a hľavne obsahuje veľké riziko omylu a prehliadnutia niektorých z účtov. Dané riziko nemusí explicitne vzísť iba z odchodu
zamestnanca z firmy. Zamestnancom sa počas pôsobenia v praxi môže meniť ich rola, projekty na ktorých pracujú, môže sa odstrániť
služba, ku ktorej musí mať prihlasovacie údaje a podobne. Všetky tieto scenáre by mali byť riešené bezpečným, jednoduchým a
centralizovaným spôsobom.

\par Jedným z riešení, ako sa vyhnúť daným rizikám je vytvorenie centrálnej autentifikácie zamestnancov. Tým by sa
zamedzilo udržiavaniu ich prihlasovacích údajov na viacerých miestach. Jednou z možností je autentifikácia pomocou ľahkého
protokolu prístupu k adresáru (Lightweight Directory Access Protocol - LDAP)~\cite{LDAP}. Cieľom je vytvorenie
servera, ktorý bude spravovať firemné prihlasovacie údaje jednotlivých zamestnancov, pričom jednotlivé služby softvérovej
firmy, ku ktorým je nutná autentifikácia budú podporovať autentifikáciu pomocou LDAP protokolu~\cite{LDAP_AUTH}. Takéto riešenie by malo
za následok správu všetkých kont zamestnancov na jednom mieste. Ak by nastali popísané rizikové situácie, likvidácia dát je umožnená
z jednoho centralizovaného bodu. Priebeh autentifikácie na firemné aplikácie a služby by prebiehal podľa schémy na
obrázku~\ref{obr_5}.

\begin{figure}[H]
\begin{center}\includegraphics[width=\textwidth,height=6cm,keepaspectratio=true]{assets/ldap_schema.png}\end{center}
\caption[Schéma prihlasovania podľa protokolu LDAP]{Schéma prihlasovania podľa protokolu LDAP}\label{obr_5}
\end{figure}

\par Treťou hrozbou bol prístup k dátam na dátovom úložisku. Je možné dôverovať osobe s autorizáciou prehliadnia dát s
citlivými informáciami ako napríklad heslá, čísla kariet, prístupové adresy a podobne? Na danú otázku existuje odpoveď
vo forme modelu CIA, známeho aj ako model rozvoja bezpečnostnej politiky~\cite{CIA}. Skratka CIA hovorí o troch dôležitých bezpečnostných
pilieroch manipulácie s dátami. Prvým z nich dôvernosť (Confidentality), ktorý hovorí o neschopnosti zistenia obsahu
dôverných informácíí ktorýmkoľvek používateľom. Ako chceme dôverné informácie ochrániť pred všetkými zamestnancami,
aj tými, ktorí majú k dátovému úložisku nutnú autorizáciu? Bezpečnými spôsobmi ochrany informácií sú enkrypcia a hašovanie.
Enkrypcia je v skratke obojsmerná funkcia. Tým pádom slúži na anonymizovanie statických dát ako napríklad informácie umožňujúce
identifikáciu osôb~\cite{Encryption}. Konkrétne sa môže jednať o dáta typu: Číslo vodičského preukazu, číslo občianskeho preukazu a podobne.
Na druhej strane hašovacia funkcia je jednosmerná. Inak povedané, funkcia nemá možnosť spätného "odhašovania" informácie~\cite{Hashing}.
Bezpečným prvkom, ktorý je k hašovaniu možné pridať je takzvaný "salt", ktorý danú hašovaciu funkciu zamieša špecifickým
reťazcom znakov. Hašovacia funkcia sa používa pri druhoch informácií, ktoré sú posielané medzi dvomi sieťovými bodmi. Napríklad
môže ísť o heslo, ktorého reálnu hodnotu nikdy na strane zariadenia ktoré ho uchováva nemusíme vedieť. Stačí, ak sa rovnaký
spôsob hašovania hesla použije na strane zariadenia vyžadujúceho autentifikáciu. Pri procese autentifikácie sa oba haše
porovnajú a ak sa rovnajú, heslá sa zhodujú. Výsledkom hašovacej funkcie je fixne dlhý reťazec znakov. Napríklad pri texte
dlhom 250 znakov môže byť výsledný haš dlhý 32 znakov. Preto dôležitým atribútom hašovacích funkcií je ich matematicky
veľmi nízka pravdepodobnosť na vytvorenie dvoch rovnakých výstupných reťazcov pri rozdielnych vstupoch. Druhým pilierom modelu
CIA je integrita (Integrity), ktorá hovorí o konzistencii informácií na dátovom úložisku. Ak by došlo k ich modifikácii, vlastník
daných informácií musí byť informovaný o ich zmene. Táto skutočnosť vie ochrániť vlastníkov údajov pred ich nežiadúcou
zmenou, ktorá môže byť spôsobená neautorizovaným prístupom. Posledným pilierom modelu CIA je dostupnosť (Availability).
K skutočnej forme informácií môže mať prístup iba ich vlastník a to v čitateľnej a konečnej podobe. To napríklad znamená,
že pred zobrazením používateľovho rodného čísla musí byť táto enkryptovaná dôverná informácia spätne dekryptovaná do
originálnej podoby.

\par Posledný spomínaný problém nesie so sebou výskyt rizík v podobe získania prístupových údajov, či už sa jedná o autentifikáciu
do firemného systému, alebo dáta potrebné na nasadenie finálneho softvérového produktu ku koncovým zariadeniam zákazníkov.
V oboch prípadoch, ako bolo už spomenuté ide o veľmi vážne bezpečnostné riziká, keďže sa taktiež priamo týkajú zákazníkov
softvérovej firmy. Pre šifrovaný spôsob komunikácie je potrebné implementovať komunikačné protokoly v súlade s bezpečnostnými
požiadavkami. Komunikácia medzi vzdialeným webovým firemným serverom a zamestnancom by hohla prebiehať cez zabezpečený protokol
HTTPS (Hypertext transfer protocol secure)~\cite{HTTPS}, kde by boli jednotlivé autentifikačné dáta na transportnej vrstve
šifrované bezpečnostným protokolom TLS (Transport Layer Security)~\cite{TLS}. Samotný spôsob autentifikácie by mohol byť
uskutočnený podľa formátu JWT (JSON Web Token) spolu s využitím autentifikačného protokolu OAUTH(2.0) (Open standard for
access delegation)~\cite{JWT}. Keďže prístup k firemným zdrojom je veľmi diskrétny, z bezpečnostného hľadiska je odporúčané
použitie dvojfaktorovej autentifikácie~\cite{DvojfaktorovaAutentifikacia}, ktorej priebeh môžme vidieť na obrázku~\ref{obr_6}.
Dvojfaktorová autentifikácia nepodlieha útokom hrubou silou a taktiež po získaní prístupových dát napríklad odchytením,
alebo inou formou útoku nepodlieha ich zneužitu z dôvodu veľmi frekventovane meniaceho sa unikátneho verifikačného kódu.
Pokiaľ sa jedná o komunikáciu medzi koncovými zariadeniami zákazníkov, tú je možné zabezpečiť protokolom SFTP (SSH File
Transfer Protocol)~\cite{SFTP}, ktorý umožňuje posielanie súborov cez protokol SSH (Secure Shell)~\cite{SSH}, slúžiaci
na bezpečné vykonávanie príkazov na vzdialených zariadeniach.

\begin{figure}[H]
\begin{center}\includegraphics[width=\textwidth,height=6cm,keepaspectratio=true]{assets/auth.png}\end{center}
\caption[Schéma dvojfaktorovej autentifikácie]{Schéma dvojfaktorovej autentifikácie}\label{obr_6}
\end{figure}

\section{Zhodnotenie}\label{sec:podsekcia-3}

\par Z analýzy vyplíva, že jednotlivé problémy spojené s bezpečnostnými rizikami sú v súčastnosti riešiteľné veľkým počtom
rôznych spôsobov protiopatrení. Taktiež existujú presne špecifikované postupy a normy, ktoré je dôležité dodržovať
softvérovými firmami pre zlepšenie ich informačnej bezpečnosti. Otázkou ostáva, z akého dôvodu naďalej väčšina malých
softvérových firiem odkladá, alebo úmyselne prehliada jednotlivé bezpečnostné hrozby. V prvom rade je dôležité si uvedomiť,
že firmy sú schopné jednotlivé bezpečnostné opatrenia vyriešiť, no z dôvodu prekážok, v podobe časových a finančných
prostriedkov im vo väčšine prípadov takáto možnosť riešenia nie je umožnená~\cite{CompanySecurity}. Príkladom môže byť nízky počet zamestnancov,
ktorý sú potrebný na jednotlivých špecializovaných úlohách  pridelených k projektom. Prevelením jednoho zamestnanca z
projektu s cieľom riešenia firemnej bezpečnosti, môže mať za následok rapídne zníženie produktivity na určitých častiach práce. Ďalším reálnym
problémom je financovanie časovo náročných bezpečnostných opatrení, ktorých implementácia je financovaná priamo z firemného
rozpočtu. Jedným z finančne a časovo nenáročných riešení pre softvérové firmy pramení z idei spojenia všetkých analyzovaných
bezpečnostných opatrení do jednoho celku, v podobe informačného bezpečnostného systému. Takýto systém by obsahoval všetky
bezpečnostné aspekty protiopatrení, pričom jeho návrh a implementácia by sa musela prioritne riadiť podľa kritérií vysokej
bezpečnosti, modularity a rozšíriteľnosti. Dôvodom je použitie implementovaného bezpečnostného riešenia v praxi jednotlivými
firmami ako produktu tretej strany, pričom firmy by mohli podľa ich potrieb jednotlivé časti systému ľahko upraviť, prípadne
doplniť o nové bezpečnostné prvky.
